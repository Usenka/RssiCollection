\chapter{Materials and Methods}

This section is to clarify the pre-existing tools, defining what was developed in this field until now, and why this tool was used instead of others.

The general structure is the following:
\begin{itemize}
	\item Definition of the specific tool(s) studied (robots, sensor nodes, smart-phones). When relevant, pre-existing experiments.
	\item Definition of the context of use (indoor/outdoor, humans/animals/robots, with/without connection).
	\item Definition of used protocols (How the data are collected, when, etc.)
\end{itemize}

\section{Wireless Sensor Networks}
	A wireless sensor networks are a collection of low-cost, low-power and multifunctional sensor nodes that are placed inside a to be monitored area   
\section{Radio Tomographic Imaging}
\section{Timeslots}
\section{Testbed}
The testbed is a wireless sensor network set up in the third floor of the SA building at the University Duisburg Essen. It covers on half of the whole floor containing a main corridor, two laboratories, two side corridors leading to three offices each, an elevator, seven smaller storage rooms and one server room. All in all the area is 531m^2.\n
To monitor the floor there are 32 nodes placed at the positions shown in figure (random number). This means the density of nodes is very low resulting in a widely spread multi-hop network. 
\subsection{Motes}
\subsection{TinyOs}




