\chapter{Discussion}

%The meaning of this paragraph is to interpret the results of the performed work. It will always connect the %introduction, the postulated hypothesis and the results of the thesis/bachelor/master.

%It should answer the following questions:
%\begin{itemize}
%	\item Could your results answer your initial questions?
%	\item Did your results agree with your initial hypothesis?
%	\item Did you close your problem, or there are still things to be solved? If yes, what will you do to %solve them? 
%\end{itemize}

\section{Possible Improvements}
The first improvement for the calibration was already explained. Here time can be saved by simply calculating the correct break and not having the system do nothing for 25 seconds.

The next improvement is for the collection. When using nodes with a lot memory it could be a good idea to not always send the data from one node directly to the sink but first collect all the data of a nodes children before forwarding it. This way the requests and data messages would take the same path as the schedule while spreading it. This would reduce the amount of requests and the over all collection time.
 
Another thing that can be improved is the algorithm that creates the schedule. Instead of just covering the network with rooted circles it is possible to later connect circles in a way that reduces the amount of messages. Also to improve choosing a next node it would be a good idea to change the calibration in a way that not only the last measured RSS for each link gets saved but a average RSS for a link.
A different way to maybe improve the schedule is to simply pick the same path a schedule message would take while spreading it. The experiments already show that spreading the schedule is faster than the rounds with the created schedule. 

\section{Conclusion}
Because the existing approach for a signal strength measurement does not work inside a multi-hop wireless sensor network a new approach was developed that works for multi-hop WSN. The new approach consists of different phases. First there is a calibration phase where information about the network is collected and paths through the WSN to collect information at a central point and to spread information inside the network are created. Then in the next phase a schedule that defines predecessors and successors for each node is created and spread inside the network. Last, the measurement of the signal strength. Therefore each node will send messages according to the schedule, meaning one node starts sending a message. All the nodes hearing that message will measure the signal strength and stores it. When a node hears a message from its predecessor it can send its own message. This goes on until every node send a message. Then the stored measurements of the nodes are collected at the central point and a new measurement round can begin.

This approach was implemented and tested. The experiments show that for a WSN with 32 widely spread nodes one measurement round takes about 500 ms followed by a 3,5 second collection phase. This provides one set of measurements every 4 seconds. 