%\setcounter{page}{2}

\cleardoublepage

\section*{Abstract}
Wireless Sensor Networks (WSN) are an emerging technology that enables new possibilities for localization and tracking. Radio Tomographic Imaging (RTI) is a localization technique that analyzes the signal strength between nodes inside a WSN, making it possible to localize and track a person without it needing to carry any device.

To do this, a system that measures the signal strengths of each link inside a WSN is needed. The approaches suggested by the literature however assume that all the nodes inside a WSN can hear each other. This makes it impossible to apply the same solution to a WSN that covers large areas like a whole floor or a whole building. To make this possible, a new approach is necessary that is able to measure the signal strength of each link in a WSN where not every node can hear all the other nodes.

This thesis provides an approach for a self calibrating signal strength measurement system that takes into account that not all the nodes can hear each other, qualifying it to be deployable in large areas. To measure the signal strength, each node needs to send a message. The suggested approach creates a schedule that defines a predecessor and successor for every node inside the WSN. On the basis of this schedule the nodes are able to send messages and measure the signal strength in a synchronized way that avoids message collisions which would result in invalid measurements for the computation of the RTI. To collect the measurement from the WSN paths from each node to the central point are defined in a calibration phase. 

The system is then tested in a WSN with 32 nodes that covers a whole floor with a size of $531 m^2$. These tests show that the suggested approach can take measurements quick enough to enable localization with RTI.
\cleardoublepage
