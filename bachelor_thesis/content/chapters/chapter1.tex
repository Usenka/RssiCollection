\chapter{Introduction}

%[You should answer the question: What is the problem?]

%This paragraph should establish the context of the reported work. To do that, authors discuss over related literature (with citations\todo{how to make %citations}\footnote{To cite a work in latex  }) and summarize the knowledge of the author in the investigated problem.

%An introduction should answer (most of) the following questions:
%\begin{itemize}
%	\item What is the problem that I want to solve?
%	\item Why is it a relevant question?
%	\item What is known before the study?
%	\item How can the study improve the current solutions?
%\end{itemize}

%To write it, use if possible active voice:
%\begin{itemize}
%	\item We are going to watch a film tonight (Active voice).
%	\item A film is going to be watched by us tonight (Passive voice).
%\end{itemize}
%The use of the first person is accepted.

%Rti need to measure the network rss values
%The existing solutions need a single hop network.
%A real deployed network can not necessary depend on this.

%What i wanted to ask you is, how much i need to write inside my motivation and problem description since i do not realy know a lot that can be written %here.
%My idea is to write about the possibility of locating with RTI. Then that this requires a system that sends messages but no two messages at one point in %time.
%Next i would explain that this is solved but only for single hop WSNs. Then give reasons for the use of it in multi-hop WSNs and say that therefore a new %method needs to be developed.
%Then i would explain the requirements for the system and thats basicaly it 


\section{Motivation}

%A good introduction usually starts presenting a general view of the topic and continues focusing on the %%problem studied. Begin it clarifying the subject area of interest and establishing the context (remember %to support it with related bibliography).

%WSN are huge at the momemnt (short explenation, ... make different stuff possible). RTI is one of these uses the rss to localize. needs a system that %sends messages. to provide high accuracy no tow messages are allowed to send at the same time. A approach that does this is needed. for single hop wsns %this is solved via multi spin. in a real world enviroment large areas can not be monitored  since wsn is not single hop.
%New approach that works in multi hop enviroment is required.

Wireless Sensor Networks (WSN) are an emerging technology that enables a lot of different applications for enviromental and habitat monitoring, analysis of structures, or localization and tracking. In the field of localizing and tracking the Radio Tomographic Imaging (RTI) was developed. It is a localization technique that analyses the signal strengths between nodes inside a WSN making it possible to localize and track a person without it carrying a device. To do this a system that measures the signal strengths between every node inside a system is needed. Such a system already exists, but it assumes that all the nodes of the WSN can hear each other to calibrate itself and to make the collection of the necessary informations possible. This limits the functionality of the system to small areas and can not be used to monitor for example a whole floor or a whole warehouse, excluding the possibility to localise in such areas. To make localization with RTI in large areas possible a new approach that makes the measurement of the signal strength in a distributed WSN possible is needed. 

\section{Problem definition}
To measure the signal strength nodes need to send messages on the basis of which the signal strength can be measured. The problem is that two messages send at the same time can collide and distort the measured signal strength. This makes procedure necessary that schedules all the nodes inside the WSN in such a way that only one node sends a message at a given point. Moreover to get a full dataset of all the signal strength the measured data from each node need to be collected at a central point making RTI localization possible. All this needs to work under the assumption that not every node is in range of all the other nodes meaning that somh

The goal of this thesis is to develop an approach that schedules the nodes message sending in a way that no two or more nodes send at the same time, measures the signal strength and collects the data at a central point to make localization with RTI possible. The system should be self calibrating and usable at any given WSN. The idea is to have a system with an initial calibration phase that defines paths to the central point from every node in the network to be able to collect the data even if a node is not directly connected to the central point. Moreover the calibration phase will define a schedule in a way that every node has a predecessor that is in its range. If the node then receives a message from its predecessor it can send its own message.   

%The goal of this thesis is to design and implement a system to measure rss of every link inside a multi hop wsn. making rti localization possible. Since %people that are localized move in a certain speed one mesurement o the whole network needs to be as fast as possible. 

\section{Bachelor Thesis Structure}
First the Material and Methods chapter will give an overview about Wireless Sensor Networks, Radio Tomographic Imaging, the existing signal strength system and a Wireless Sensor Network that is located at the University Duisburg-Essen and used to test the developed approach. Then the Approach chapter explains the developed approach for a signal strength measurement system working in every WSN is explained. Next int the Implementation chapter it will be explained how the developed approach can be implemented. After that, in the Evaluation chapter, the developed system will be evaluated. Lastly the Discussion chapter will discuss the approach.