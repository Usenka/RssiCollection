\chapter{Implementation}

In this chapter it will be explained how to implement a system described in Chapter 3. 

\section{General Structure}
The base station is a computer that runs a Java application. On the nodes runs code written in NesC for TinyOS.
\subsection{General Base Station}
The Base Station has two main tasks. First it needs to collect and store the data for further processes.
Secondly the schedule needs to be created and send to the sink.  
\subsection{General mote Application}
The application running on the motes has multiple tasks. It needs to be able to calibrate the network, collect data from the network, spread the schedule inside the network and sample the RSSI. Therefore it is split into multiple modules with fitting interfaces that each handle exactly one task. The structure of the application is shown in Figure x.x (Awesome diagram :))   

\section{Base Station}
\subsection{Receiving and Storing the Data}

\subsection{Creating the Schedule}
\section{Calibration}
\section{Collection}
\section{Creating the Schedule}
\section{Spreading the Schedule}
\section{Sampling}